\documentclass{article}

\usepackage{listings}

\title{The Language Grammar}
\author{Mikolaj Walczak}

\begin{document}
\maketitle
\clearpage

\section{The lexical structure of Grammar}

\subsection{Identifiers}

Identifiers \textit{Ident} are unquoted strings beginning with a letter,
followed by any combination of letters, digits, and the characters \texttt{\_ '}
reserved words excluded.

\subsection{Literals}

Integer literals \textit{Integer} are nonempty sequences of digits.

String literals \textit{String} have the form
\texttt{"}\textit{x}\texttt{"}\}, where \textit{x} is any sequence of any characters
except \texttt{"} unless preceded by \texttt{$\backslash$}.

Error literals \textit{Error} have the form
\texttt{`}\textit{x}\texttt{`}\}, where \textit{x} is any sequence of any characters
except \texttt{`} unless preceded by \texttt{$\backslash$}.

\subsection{Reserved words and symbols}

The set of reserved words is the set of terminals appearing in the grammar.
Those reserved words that consist of non-letter characters are called symbols,
and they are treated in a different way from those that are similar to
identifiers. The lexer follows rules familiar from languages like Haskell,
C, and Java, including longest match and spacing conventions.

The reserved words used in Grammar are the following:

\begin{center}\begin{tabular}{llll}
\texttt{None} & \texttt{and} & \texttt{any} & \texttt{bool} \\
\texttt{break} & \texttt{continue} & \texttt{elif} & \texttt{else} \\
\texttt{error} & \texttt{false} & \texttt{func} & \texttt{if} \\
\texttt{int} & \texttt{not} & \texttt{or} & \texttt{print} \\
\texttt{return} & \texttt{string} & \texttt{true} & \texttt{var} \\
\texttt{void} & \texttt{while} &  \\
\end{tabular}\end{center}

The symbols used in Grammar are the following:

\begin{center}\begin{tabular}{llll}
[ & ] & $<$$|$ & $|$$>$ \\
$<$ & $>$ & $<$= & $>$= \\
== & * & / & \% \\
+ & - & ( & ) \\
, & . & -$>$ & = \\
; & \{ & \} & $<$- \\
\end{tabular}\end{center}

\subsection{Comments}

Single-line comments begin with //.Multiple-line comments are  enclosed with /* and */.

\section{The syntactic structure of Grammar}

Non-terminals are enclosed between $<$ and $>$.
The symbols -$>$ (production),  \textbf{$|$}  (union)
and \textbf{eps} (empty rule) belong to the BNF notation.
All other symbols are terminals. \\
\begin{center}\begin{tabular}{lll}
\textit{Program} & -$>$ & \textit{[Decl]} \textit{[Stmt]} \\
\textit{Boolean} & -$>$ & \texttt{true} \\
 & \multicolumn{1}{c}{\textbf{$|$}} & \texttt{false} \\
\textit{Literal} & -$>$ & \texttt{None} \\
 & \multicolumn{1}{c}{\textbf{$|$}} & \textit{Integer} \\
 & \multicolumn{1}{c}{\textbf{$|$}} & \textit{Boolean} \\
 & \multicolumn{1}{c}{\textbf{$|$}} & \textit{String} \\
 & \multicolumn{1}{c}{\textbf{$|$}} & \textit{TokenError} \\
 & \multicolumn{1}{c}{\textbf{$|$}} & \texttt{[} \textit{[Exp]} \texttt{]} \\
 & \multicolumn{1}{c}{\textbf{$|$}} & \texttt{$<$$|$} \textit{[Exp]} \texttt{$|$$>$} \\
\textit{Comp} & -$>$ & \texttt{$<$} \\
 & \multicolumn{1}{c}{\textbf{$|$}} & \texttt{$>$} \\
 & \multicolumn{1}{c}{\textbf{$|$}} & \texttt{$<$=} \\
 & \multicolumn{1}{c}{\textbf{$|$}} & \texttt{$>$=} \\
 & \multicolumn{1}{c}{\textbf{$|$}} & \texttt{==} \\
\textit{BOp} & -$>$ & \texttt{and} \\
 & \multicolumn{1}{c}{\textbf{$|$}} & \texttt{or} \\
\textit{Exp6} & -$>$ & \textit{Call} \\
 & \multicolumn{1}{c}{\textbf{$|$}} & \textit{Ident} \\
 & \multicolumn{1}{c}{\textbf{$|$}} & \textit{Literal} \\
 & \multicolumn{1}{c}{\textbf{$|$}} & \texttt{(} \textit{Exp} \texttt{)} \\
\textit{Exp5} & -$>$ & \textit{Exp5} \texttt{*} \textit{Exp6} \\
 & \multicolumn{1}{c}{\textbf{$|$}} & \textit{Exp5} \texttt{/} \textit{Exp6} \\
 & \multicolumn{1}{c}{\textbf{$|$}} & \textit{Exp5} \texttt{\%} \textit{Exp6} \\
 & \multicolumn{1}{c}{\textbf{$|$}} & \textit{Exp6} \\
\textit{Exp4} & -$>$ & \textit{Exp4} \texttt{+} \textit{Exp5} \\
 & \multicolumn{1}{c}{\textbf{$|$}} & \textit{Exp4} \texttt{-} \textit{Exp5} \\
 & \multicolumn{1}{c}{\textbf{$|$}} & \textit{Exp5} \\
\textit{Exp3} & -$>$ & \textit{Exp3} \textit{Comp} \textit{Exp4} \\
 & \multicolumn{1}{c}{\textbf{$|$}} & \textit{Exp4} \\
\textit{Exp2} & -$>$ & \texttt{not} \textit{Exp3} \\
 & \multicolumn{1}{c}{\textbf{$|$}} & \textit{Exp3} \\
\textit{Exp1} & -$>$ & \textit{Exp1} \textit{BOp} \textit{Exp2} \\
 & \multicolumn{1}{c}{\textbf{$|$}} & \textit{Exp2} \\
\textit{Exp} & -$>$ & \textit{Exp1} \\
\textit{[Exp]} & -$>$ & \textbf{eps} \\
 & \multicolumn{1}{c}{\textbf{$|$}} & \textit{Exp} \\
 & \multicolumn{1}{c}{\textbf{$|$}} & \textit{Exp} \texttt{,} \textit{[Exp]} \\
\textit{Call} & -$>$ & \textit{Ident} \texttt{(} \textit{[Exp]} \texttt{)} \\
 & \multicolumn{1}{c}{\textbf{$|$}} & \textit{Ident} \texttt{.} \textit{Ident} \texttt{(} \textit{[Exp]} \texttt{)} \\
\end{tabular}\end{center}
\begin{center}\begin{tabular}{lll}
\textit{Type} & -$>$ & \texttt{any} \\
 & \multicolumn{1}{c}{\textbf{$|$}} & \texttt{void} \\
 & \multicolumn{1}{c}{\textbf{$|$}} & \texttt{int} \\
 & \multicolumn{1}{c}{\textbf{$|$}} & \texttt{bool} \\
 & \multicolumn{1}{c}{\textbf{$|$}} & \texttt{string} \\
 & \multicolumn{1}{c}{\textbf{$|$}} & \texttt{error} \\
 & \multicolumn{1}{c}{\textbf{$|$}} & \texttt{[} \textit{Type} \texttt{]} \\
 & \multicolumn{1}{c}{\textbf{$|$}} & \texttt{$<$$|$} \textit{[Type]} \texttt{$|$$>$} \\
 & \multicolumn{1}{c}{\textbf{$|$}} & \texttt{(} \textit{[Type]} \texttt{)} \texttt{-$>$} \textit{Type} \\
\textit{[Type]} & -$>$ & \textit{Type} \\
 & \multicolumn{1}{c}{\textbf{$|$}} & \textit{Type} \texttt{,} \textit{[Type]} \\
\textit{Param} & -$>$ & \textit{Ident} \textit{Type} \\
\textit{[Param]} & -$>$ & \textbf{eps} \\
 & \multicolumn{1}{c}{\textbf{$|$}} & \textit{Param} \\
 & \multicolumn{1}{c}{\textbf{$|$}} & \textit{Param} \texttt{,} \textit{[Param]} \\
\textit{[Ident]} & -$>$ & \textbf{eps} \\
 & \multicolumn{1}{c}{\textbf{$|$}} & \textit{Ident} \\
 & \multicolumn{1}{c}{\textbf{$|$}} & \textit{Ident} \texttt{,} \textit{[Ident]} \\
\textit{Decl} & -$>$ & \texttt{var} \textit{Ident} \textit{Type} \texttt{=} \textit{Exp} \\
 & \multicolumn{1}{c}{\textbf{$|$}} & \texttt{func} \textit{Ident} \texttt{(} \textit{[Param]} \texttt{)} \texttt{-$>$} \textit{Type} \textit{Stmt} \\
\textit{[Decl]} & -$>$ & \textbf{eps} \\
 & \multicolumn{1}{c}{\textbf{$|$}} & \textit{Decl} \\
 & \multicolumn{1}{c}{\textbf{$|$}} & \textit{Decl} \texttt{;} \textit{[Decl]} \\
\textit{Elif} & -$>$ & \texttt{elif} \textit{Exp} \textit{Stmt} \\
\textit{[Elif]} & -$>$ & \textbf{eps} \\
 & \multicolumn{1}{c}{\textbf{$|$}} & \textit{Elif} \textit{[Elif]} \\
\textit{Stmt} & -$>$ & \texttt{\{} \textit{[Decl]} \textit{[Stmt]} \texttt{\}} \\
 & \multicolumn{1}{c}{\textbf{$|$}} & \textit{Ident} \texttt{=} \textit{Exp} \\
 & \multicolumn{1}{c}{\textbf{$|$}} & \textit{[Ident]} \texttt{$<$-} \textit{Exp} \\
 & \multicolumn{1}{c}{\textbf{$|$}} & \texttt{return} \textit{Exp} \\
 & \multicolumn{1}{c}{\textbf{$|$}} & \texttt{print} \textit{[Exp]} \\
 & \multicolumn{1}{c}{\textbf{$|$}} & \texttt{if} \textit{Exp} \textit{Stmt} \textit{[Elif]} \\
 & \multicolumn{1}{c}{\textbf{$|$}} & \texttt{if} \textit{Exp} \textit{Stmt} \textit{[Elif]} \texttt{else} \textit{Stmt} \\
 & \multicolumn{1}{c}{\textbf{$|$}} & \texttt{while} \textit{Exp} \textit{Stmt} \\
 & \multicolumn{1}{c}{\textbf{$|$}} & \texttt{break} \\
 & \multicolumn{1}{c}{\textbf{$|$}} & \texttt{continue} \\
 & \multicolumn{1}{c}{\textbf{$|$}} & \textit{Call} \\
\textit{[Stmt]} & -$>$ & \textbf{eps} \\
 & \multicolumn{1}{c}{\textbf{$|$}} & \textit{Stmt} \texttt{;} \textit{[Stmt]} \\
\end{tabular}\end{center}

\clearpage

\section{Language description}

\subsection{Definitions and Assignments}
\label{def}
\begin{itemize}
  \setlength\itemsep{.1em}
  \item $T$ --- a type \textbf{Annotation}. $T$ can be of any proper type
  available in the language. This excludes the usage of $TNone$ unless
  as a return type for a function and $TAny$, used internally to indicate the type
  of an empty array \texttt{[]}.
  \item $ident$ --- a variable identifier.
\end{itemize}

\subsubsection{Variables}
\label{variables}
\texttt{var $ident$ $T$ = $value$; } --- will declare a variable $ident$ of
type~$T$ and assign it a value of $value$. All variables must be intialized
during declaration to avoid unintialized variables access. \\
\texttt{$ident$ = $value$; } --- will assign $ident$ an evaluated $value$. When
the type mismatch an exception will be raised during typechecking, before the
program execution.

\subsubsection{Functions}
\label{def-functions}
\begin{itemize}
  \setlength\itemsep{.1em}
  \item return type $R$ can be any proper type as stated in \ref{variables}, but
    allows for usage of $TNone$ to indicate a procedure.
  \item the number of parameters $k \geq 0$.
  \item parameter identifiers $p_i$ of type $T_i$ where $i = 1..k$.
  \item a function body $S$.
\end{itemize}

\texttt{func $ident$($p_1$ $T_1$, ..., $p_k$ $T_k$) -> $R$ $S$; } --- will
define a function $ident$ with $k$ parameters, a return type $R$ and a body $S$.

Functions can be passed and returned from another functions for example: \\
\texttt{func apply(x int, f (int) -> int) -> int return f(x); }

\subsubsection{Basic types}

\begin{table}[h]
\centering
\label{primitive-types}
\begin{tabular}{|l|l|l|l|l|} \hline
\textbf{Type} & \textbf{Annotation} & \textbf{Value literal}               & \textbf{Example}   \\ \hline
Integer       & \texttt{int}        & \texttt{-?\textbackslash d+}         & \texttt{-42}       \\ \hline
String        & \texttt{string}     & \texttt{"(\textbackslash c - '"')*"} & \texttt{"example"} \\ \hline
Error         & \texttt{error}      & \texttt{`(\textbackslash c - '`')*`} & \texttt{`example`} \\ \hline
Boolean       & \texttt{bool}       & \texttt{(true|false)}                & \texttt{true}      \\ \hline
\end{tabular}
\caption{Primitive types}
\end{table}

\subsubsection{Container types}
$T$ represents a type \textbf{Annotation}. $t$ represents a \textbf{Value}
expression for the type~$T$.

\begin{table}[h]
\centering
\label{container-types}
\begin{tabular}{|l|l|l|l|l|} \hline
\textbf{Type} & \textbf{Annotation}                   & \textbf{Value literal}                      \\ \hline
Array         & \texttt{[$T$]}                        & \texttt{[($t$(, $t$)*)?]}                   \\ \hline
Map           & \texttt{\{$T_1$:$T_2$\}}              & \texttt{\{($t_1$:$t_2$(, $t_1$:$t_2$)*)?\}} \\ \hline
Tuple         & \texttt{<|$T_1$, $T_2$, ..., $T_k$|>} & \texttt{<|$t_1$, $t_2$, ..., $t_k$|>}       \\ \hline
Function      & \texttt{($T_1$, $T_2$, ..., $T_k$) -> $R$} & \textit{Access only by an identifier}  \\ \hline
\end{tabular}
\caption{Container types}
\end{table}

\paragraph{Array examples}
\texttt{var arr [int] = [];} \\
\texttt{var arr [int] = [0];} \\
\texttt{var arr [int] = [0, 42];} \\
\texttt{var arr [string] = ["this", "is", "an", "example"];}

\paragraph{Tuple examples}
\texttt{var tuple <|int, int|> = <|0, 0|>;} \\
\texttt{var tuple <|string, int, int|> = <|"example", 2, 42|>;} \\
\texttt{var tuple <|int, string, int, int|> = <|-2, "example", 1, 2|>;} \\
\texttt{var tuple <|int, <|int, int|>|> = <|-2, <|1, 2|>|>;}

\paragraph{Tuple unpacking}
\texttt{var x int = 0;} \\
\texttt{var y string = "";} \\
\texttt{var t <|int,string|> = <|42, "John"|>;} \\
\texttt{x, y <- t; } \\
will assign $x$ the value \texttt{42} and $y$ the value \texttt{"John"}.

\subsection{Expressions}
Expressions should be generally considered as an rvalue.

\subsubsection{Arithmetic expressions}
Supports standard arithmetic operations $+, -, *, /, \%, ()$ as well as
variables and function calls (if the return type is an \texttt{int}). Arithmetic
expressions can have side effects (a function call). Requires both parameters to
be integers.

\paragraph{Examples}
\texttt{2} \\
\texttt{2 + 4 * (4 + 1)} \\
\texttt{2 + x * (y - 1)} \\
\texttt{2 + power(x, 2)}

\subsubsection{Boolean expressions}
Supports standard boolean operations $and, or, not, ()$, arithmetic expressions
comparison $<, >, <=, >=, ==$ as well as variables and function calls (if the
return type is a \texttt{bool}). Boolean expressions can have side effects
(a function call).

\paragraph{Examples}
\texttt{true} \\
\texttt{true or (true and false)} \\
\texttt{(p and not q) or isEven(x)} \\
\texttt{x / 2 == 3 or x < 10}

\subsubsection{Others}
Expressions might also be literals such as \ref{primitive-types} and
\ref{container-types} as well as function calls and even function references.

\paragraph{Examples}
\texttt{[1, 2, 3]} \\
\texttt{<|x, 32 * 123|>} \\
\texttt{myFunc(2 * x - 7, [])} \\
\texttt{myFunc}

\subsection{Statements}
All statements must end with a \texttt{;} symbol. We will use \texttt{[STMT]} to
designate a non empty list of statements.

\subsubsection{Definitions and assignments}
As described in \ref{def}.

\subsubsection{Conditional execution}
Standard \texttt{if .. then .. elif .. else}.

\subsubsection{While loop}
Standard \texttt{while} with \texttt{break} and \texttt{continue} instructions.

\subsubsection{Function calls}
Functions can be called like \texttt{isEven(x)}. Types also have builtin
methods which can be called like \texttt{x.ToString()} or \texttt{arr.Length()}.

\subsubsection{Return}
Functions can return value with \texttt{return} keyword. Unless a function is
of type $(_) -> TNone$ a runtime error will be thrown if the function ends
without returning a value. Calling \texttt{return} outside of function body
declaration will also result in a runtime error.

\subsubsection{Print}
An instructions which prints given value to the screen. The parameters can be
separated by a ,,\textit{,}'' to achieve a concatenation. For example: \\
\texttt{var name string = "Vincent"; } \\
\texttt{print "Hello ", name, "!"; } \\
will print the string \textit{Hello Vincent!} to the screen.

\section{Summary}
Language satisfies the following requirements (24points):
\begin{itemize}
  \item multiple types (1.)
  \item expressions (2.)
  \item while, if..elif..else (3.)
  \item functions with recrusion (4.)
  \item print (5.)
  \item string and type conversion functions (6c.) expressions side-efects (6d.)
  \item static typing (7.)
  \item static variable binding (8.)
  \item runtime errors (9.)
  \item functions with return values (10.)
  \item arrays (11b.) and nested tuples (11d.)
  \item function nesting with static binding (12.)
  \item break-continue (11e.), functions as parameters (11f.), returning function
    closures (11g.)
\end{itemize}

\end{document}
